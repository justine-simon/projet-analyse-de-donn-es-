\documentclass[11pt]{beamer}

% ----------------------------
% Thème (au choix)
% ----------------------------
\usetheme{Madrid}      

% ----------------------------
% Packages utiles
% ----------------------------
\usepackage[utf8]{inputenc}
\usepackage[T1]{fontenc}
\usepackage[french]{babel}
\usepackage{graphicx}
\usepackage{amsmath}

% ----------------------------
% slide 1
% ----------------------------
\title{analyse de données}
\subtitle{Optimisation des karts sur MarioKart 8 Deluxe}
\author{Justine Simon Clarisse Le Philippe}
\institute{Université de Strasbourg}
\date{\today}


\begin{document}

% Page de titre
\begin{frame}
  \titlepage
\end{frame}

% ----------------------------
% slide 2
% ----------------------------
\begin{frame}{Contexte du projet }
Sur Mario Kart 8 Deluxe, le joueur peur choisir entre plusieurs composants pour son kart : 
\begin{itemize}
    \item le personnage
    \item le kart
    \item les roues, 
    \item le planeur, 
\end{itemize}

\vspace{0.3cm}
Chaque composant a des statistiques spécifiques :
\begin{itemize}
    \item la vitesse, 
    \item l’accélération, 
    \item la maniabilité, 
    \item le poids...  
\end{itemize}
\vspace{0.3cm}

Le choix d’un véhicule repose sur des arbitrages entre ces statistiques, qui dépendent du style de jeu du joueur et du type de circuit sélectionné.

\end{frame}

% ----------------------------
% slide 3
% ----------------------------

\begin{frame}{Objectifs du projet}
\textbf {\underline{L’objectif global} : automatiser le choix de la meilleure combinaison possible en fonction de ces éléments, afin de proposer une aide à la décision personnalisée.}

\vspace{0.6cm}
\underline{Sous objectifs} :
\begin{itemize}
    \item intégrer des préférences utilisateur via un questionnaire interactif
    \item mettre en place un système de pondération contextuelle
    \item  automatiser le classement et la sélection de solutions optimales
    \item  proposer une aide à la décision personnalisée pour le joueur

\end{itemize}
\end{frame}


% ----------------------------
% slide 4
% ----------------------------

\begin{frame}{Explication des scores}
Création de bases de données à partir des statistiques du site Mario Wiki

\vspace{0.3cm}
Attribution de points à chaque composant du kart et à chaque personnage pour chaque statistiques prises en compte :
\begin{itemize}
    \item vitesse (sol, eau, air, antigravité)
    \item maniabilité / manutention (sol, eau, air, antigravité)
    \item poids
    \item accélération
    \item mini-turbo
\end{itemize}

\vspace{0.3cm}
\underline{Principe des points} :

Le score total est exprimé en point et correspond à la somme des points du personnage, du kart, des roues et du planeur, qui est compris entre 0 et 20

L’objectif étant d’obtenir la valeur la plus élevée possible afin d’être le plus performant.
\end{frame}

% ----------------------------
% slide 5
% ----------------------------
\begin{frame}{Construction et préparation des bases de données}
Construction de quatre bases de données : personnages, karts, roues, planeurs

\vspace{0.3cm}

\begin{center}
\includegraphics[width=0.9\textwidth]{image/tableau.png}
\end{center}


\begin{itemize}
  \item nettoyage automatique des noms de colonnes
  \item Gestion du séparateur décimal
  \item correction des valeurs manquantes (variable taille des personnages)
  \item calcul de statistiques moyennes de vitesse et de maniabilité selon l’environnement
 \end{itemize} 

\end{frame}


% ----------------------------
% slide 6
% ----------------------------

\begin{frame}{Classification des circuits}

Regroupement des circuits en 4 catégories :
  \begin{itemize}
    \item \texttt{EAU}
    \item \texttt{VOL}
    \item \texttt{VITESSE}
    \item \texttt{TECHNIQUE}
  \end{itemize}

Chaque circuit appartient à une seule catégorie basée sur l'environnement dominant

 \vspace{0.3cm}
Pondérations spécifiques selon le type de circuit

  \underline{Exemple} : circuit aquatique :
  
\begin{center}

\texttt{sol} : 40 \%\\
\texttt{eau} : 45 \%\\
\texttt{air} : 5 \%\\
\texttt{antigravité} : 10 \%
\end{center}


\end{frame}



\end{document}
