\documentclass[11pt]{beamer}

% ----------------------------
% Thème (au choix)
% ----------------------------
\usetheme{Madrid}      

% ----------------------------
% Packages utiles
% ----------------------------
\usepackage[utf8]{inputenc}
\usepackage[T1]{fontenc}
\usepackage[french]{babel}
\usepackage{graphicx}
\usepackage{amsmath}

\usepackage[T1]{fontenc}
\usepackage[utf8]{inputenc} % si Overleaf: ok
\usepackage[french]{babel}
\usepackage{amsmath}
\usepackage{graphicx}
\usepackage{xcolor}

% Code formatting
\usepackage{listings}
\lstdefinestyle{Rstyle}{
  language=R,
  basicstyle=\ttfamily\small,
  keywordstyle=\color{blue},
  commentstyle=\color{gray},
  stringstyle=\color{teal},
  breaklines=true,
  showstringspaces=false,
  frame=single,
  rulecolor=\color{black},
  tabsize=2
}

% ----------------------------
% slide 1
% ----------------------------
\title{Analyse de données}
\subtitle{Optimisation des karts sur MarioKart 8 Deluxe}
\author{Justine Simon Clarisse Le Philippe}
\institute{Université de Strasbourg}
\date{\today}


\begin{document}

% Page de titre
\begin{frame}
  \titlepage
\end{frame}

% ----------------------------
% slide 2
% ----------------------------
\begin{frame}{Contexte du projet }
Sur Mario Kart 8 Deluxe, le joueur peur choisir entre plusieurs composants pour son kart : 
\begin{itemize}
    \item le personnage
    \item le kart
    \item les roues, 
    \item le planeur, 
\end{itemize}

\vspace{0.3cm}
Chaque composant a des statistiques spécifiques :
\begin{itemize}
    \item la vitesse, 
    \item l’accélération, 
    \item la maniabilité, 
    \item le poids...  
\end{itemize}
\vspace{0.3cm}

Le choix d’un véhicule repose sur des arbitrages entre ces statistiques, qui dépendent du style de jeu du joueur et du type de circuit sélectionné.

\end{frame}

% ----------------------------
% slide 3
% ----------------------------

\begin{frame}{Objectifs du projet}
\textbf {\underline{L’objectif global} : automatiser le choix de la meilleure combinaison possible en fonction de ces éléments, afin de proposer une aide à la décision personnalisée.}

\vspace{0.6cm}
\underline{Sous objectifs} :
\begin{itemize}
    \item Intégrer des préférences utilisateur via un questionnaire interactif.
    \item Mettre en place un système de pondération contextuelle.
    \item  Automatiser le classement et la sélection de solutions optimales.
    \item  Proposer une aide à la décision personnalisée pour le joueur.

\end{itemize}
\end{frame}


% ----------------------------
% slide 4
% ----------------------------

\begin{frame}{Explication des scores}
Création de bases de données à partir des statistiques du site Mario Wiki.

\vspace{0.3cm}
Attribution de points à chaque composant du kart et à chaque personnage pour chaque statistiques prises en compte :
\begin{itemize}
    \item vitesse (sol, eau, air, antigravité)
    \item maniabilité / manutention (sol, eau, air, antigravité)
    \item poids
    \item accélération
    \item mini-turbo
\end{itemize}

\vspace{0.3cm}
\underline{Principe des points} :

Le score total est exprimé en point et correspond à la somme des points du personnage, du kart, des roues et du planeur, qui est compris entre 0 et 20.

L’objectif étant d’obtenir la valeur la plus élevée possible afin d’être le plus performant.
\end{frame}

% ----------------------------
% slide 5
% ----------------------------
\begin{frame}{Construction et préparation des bases de données}
Construction de quatre bases de données : personnages, karts, roues, planeurs.

\vspace{0.3cm}

\begin{center}
\includegraphics[width=0.9\textwidth]{image/tableau.png}
\end{center}


\begin{itemize}
  \item Nettoyage automatique des noms de colonnes.
  \item Gestion du séparateur décimal.
  \item Correction des valeurs manquantes (variable taille des personnages).
  \item Calcul de statistiques moyennes de vitesse et de maniabilité selon l’environnement.
 \end{itemize} 

\end{frame}


% ----------------------------
% slide 6
% ----------------------------

\begin{frame}{Classification des circuits}

Regroupement des circuits en 4 catégories :
  \begin{itemize}
    \item \texttt{EAU}
    \item \texttt{VOL}
    \item \texttt{VITESSE}
    \item \texttt{TECHNIQUE}
  \end{itemize}

Chaque circuit appartient à une seule catégorie basée sur l'environnement dominant.

 \vspace{0.3cm}
Pondérations spécifiques selon le type de circuit.

  \underline{Exemple} : Circuit aquatique :
  
\begin{center}

\texttt{sol} : 40 \%\\
\texttt{eau} : 45 \%\\
\texttt{air} : 5 \%\\
\texttt{antigravité} : 10 \%
\end{center}


\end{frame}

\begin{frame}{Questionnaire: Choix du personnage et du circuit}

\textbf{Étape 1 — Choix du personnage}

\begin{itemize}
  \item  L'utilisateur saisit le nom de son personnage. 
  \item[$>$] Le programme transforme automatiquement la saisie et la base en minuscules et supprime les espaces inutiles.
  \item Une boucle vérifie si le personnage existe réellement dans la base.
  \item[$>$] Si le personnage est introuvable, la liste complète est réaffichée et une nouvelle saisie est demandée.
\end{itemize}

\vspace{0.3cm}

\textbf{Étape 2 — Choix du circuit}

\begin{itemize}
  \item L'utilisateur saisit le nom du circuit.
  \item Le programme nettoie également cette saisie (minuscules et suppression des espaces).
  \item Une boucle vérifie l'existence du circuit.
  \item[$>$] En cas d’erreur, la liste des circuits est réaffichée et la saisie est redemandée.
\end{itemize}

\end{frame}

% =========================
% 5) Normalisation
% =========================
\begin{frame}{Questionnaire : préférences et normalisation}
\begin{itemize}
  \item L'utilisateur note 5 critères entre 0 et 10 : vitesse, mini-turbo, maniabilité, accélération et poids.
  \item \textbf{Pourquoi normaliser ?} Pour interpréter les notes comme des \textbf{poids relatifs} comparables entre utilisateurs.
  \item Exemple : \texttt{10,10,10,10,10} $\Rightarrow$ chaque critère vaut \textbf{20\%} $\Rightarrow$ il veut un kart polyvalent.
\end{itemize}

\vspace{0.15cm}
\begin{lstlisting}[style=Rstyle]
prefs <- c(vitesse=p_vitesse, mini_turbo=p_drift,
           maniabilite=p_mania, acceleration=p_accel, poids=p_poids)

if (sum(prefs) == 0) prefs <- prefs + 1
#prefs <- prefs / sum(prefs)   # somme = 1
\end{lstlisting}
\begin{block}{Principe de la normalisation}
La normalisation consiste à diviser chaque préférence par la somme totale afin d'obtenir un vecteur dont la somme vaut 1.
\end{block}
\end{frame}

% =========================
% 6) Combos et addition stats
% =========================
\begin{frame}{Algorithme : combinaisons et addition des statistiques}
\begin{itemize}
  \item On génère toutes les combinaisons possibles \textbf{kart $\times$ roue $\times$ planeur} (produit cartésien).
  \item Problème : mêmes noms de colonnes (\texttt{vitesse}, \texttt{poids}, etc.) dans chaque table.
  \item Solution : suffixer les colonnes (\texttt{\_kart}, \texttt{\_roue}, \texttt{\_planeur}), puis additionner.
\end{itemize}

\begin{lstlisting}[style=Rstyle]
combos <- crossing(
  kart %>% select(nom_kart, all_of(stats_cols)) %>%
    rename_with(~ paste0(.x, "_kart"), all_of(stats_cols)),
  roue %>% select(nom_roue, all_of(stats_cols)) %>%
    rename_with(~ paste0(.x, "_roue"), all_of(stats_cols)),
  planeur %>% select(nom_planeur, all_of(stats_cols)) %>%
    rename_with(~ paste0(.x, "_planeur"), all_of(stats_cols))
)

for (col in stats_cols) {
  full[[col]] <- full[[paste0(col,"_kart")]] +
                 full[[paste0(col,"_roue")]] +
                 full[[paste0(col,"_planeur")]] +
                 p[[col]]   # personnage
}
\end{lstlisting}
\end{frame}
% =========================
% Algorithme : Combinaisons et addition
% =========================
\begin{frame}[fragile]{Algorithme (1) : combinaisons et addition des statistiques}

\begin{itemize}
  \item Génération de toutes les combinaisons \textbf{kart $\times$ roue $\times$ planeur} (produit cartésien).
  \item[$>$] Problème : colonnes dupliquées (\texttt{vitesse}, \texttt{poids}, etc.) dans chaque table.
  \item[$>$] Solution : ajout de suffixes (\texttt{\_kart}, \texttt{\_roue}, \texttt{\_planeur}) pour distinguer l’origine.
  \item Addition finale : \textbf{statistiques pièces + statistiques personnage}.
\end{itemize}

\vspace{0.2cm}

{\tiny
\begin{lstlisting}[style=Rstyle,basicstyle=\ttfamily\tiny,breaklines=true]
combos <- crossing(
  kart %>% select(nom_kart, all_of(stats_cols)) %>% rename_with(~paste0(.x,"_kart"), all_of(stats_cols)),
  roue %>% select(nom_roue, all_of(stats_cols)) %>% rename_with(~paste0(.x,"_roue"), all_of(stats_cols)),
  planeur %>% select(nom_planeur, all_of(stats_cols)) %>% rename_with(~paste0(.x,"_planeur"), all_of(stats_cols))
)

for (col in stats_cols) {
  full[[col]] <- full[[paste0(col,"_kart")]] +
                 full[[paste0(col,"_roue")]] +
                 full[[paste0(col,"_planeur")]] +
                 p[[col]]
}
\end{lstlisting}
}

\end{frame}



% =========================
% 7) Pondérations circuit : v_ctx, m_ctx
% =========================
\begin{frame}{Algorithme (2) : adaptation au circuit}
\begin{itemize}
  \item Les circuits sont classés en catégories :Eau, Vol, Vitesse, Technique.
  \item Chaque catégorie fournit des pondérations $w=(w_{sol}, w_{eau}, w_{air}, w_{anti})$.
  \item On calcule ensuite :
  \begin{itemize}
    \item \textbf{vitesse contextuelle} $v_{ctx}$ (pondérée selon l’environnement)
    \item \textbf{maniabilité contextuelle} $m_{ctx}$
  \end{itemize}
\end{itemize}
\end{frame}

% 8) Score & classement
% =========================
\begin{frame}[fragile]{Algorithme (3) : score global et classement}

\begin{block}{Score final}
Somme pondérée des critères, selon les préférences normalisées.
\end{block}

\begin{itemize}
  \item Le programme affiche :
  \begin{itemize}
    \item le \textbf{meilleur combo} (score maximal),
    \item un \textbf{Top 10} des meilleures alternatives.
  \end{itemize}
  \item Intérêt : proposer un choix optimal mais aussi des options proches.
\end{itemize}

{\tiny
\begin{lstlisting}[style=Rstyle]
cat("=== MEILLEUR COMBO ===\n")
print(resultats %>% slice(1) %>%
  select(nom_kart, nom_roue, nom_planeur, score))

cat("=== TOP N ===\n")
print(resultats %>% slice_head(n = top_n) %>%
  select(nom_kart, nom_roue, nom_planeur, score))
\end{lstlisting}
}

\end{frame}




\end{document}
